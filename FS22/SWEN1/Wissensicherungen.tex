\documentclass{article}

% Language setting
% Replace `english' with e.g. `spanish' to change the document language
\usepackage[german]{babel}

% Set page size and margins
% Replace `letterpaper' with `a4paper' for UK/EU standard size
\usepackage[a4paper]{geometry}

% Useful packages
\usepackage{amsmath}
\usepackage{graphicx}
\usepackage[colorlinks=true, allcolors=blue]{hyperref}
\usepackage{float}
\usepackage{multicol}
\usepackage{xcolor}

\setlength{\columnseprule}{0.5pt}
\def\columnseprulecolor{\color{blue}}
\title{Wissensicherung}
\author{Asha Schwegler}

\begin{document}
\maketitle

\tableofcontents

\pagebreak

\section{LE01}
\subsection{Was ist Software Engineering?}

\begin{itemize}
 \item Herstellung oder Entwicklung 
von Software,  Organisation und Modellierung der zugehörigen Datenstrukturen und 
dem Betrieb von Softwaresystemen. 
 \item Anhand eines strukturierten (Projekt-)Planes. (Schritte, Phasen, Meilensteine)
 \item Schritte während Entw.Prozess eng miteinander verzahnt.
\end{itemize}

\subsection{Was für Prozesse bzw. Disziplinen können im Software Engineering unterschieden werden?}

\paragraph{Kernprozesse\\}
\begin{itemize}
	\item Anforderungserhebung
	\item Systemdesign/technische Konzeption
	\item Implementierung
	\item Softwaretest
	\item Softwareeinführung
	\item Wartung/Pflege
\end{itemize}

\paragraph{Unterstützungsprozesse\\}
\begin{itemize}
	\item Projektmanagement
	\item Qualitätsmanagement
	\item Risikomanagement
\end{itemize}

\subsection{Was sind die Charakteristiken eines iterativ-inkrementellen Softwarenentwicklungsprozesses?}
\begin{itemize}
	\item Abwicklung in Iterationen
	\item Inkrement = In jeder Iteration ein Stück SW entwickelt
	\item Ziele sind Risiko-getrieben
	\item Iterationsreviews mit Learnings für nächste Iteration
\end{itemize}



\subsection{Warum wird im Software Engineering modelliert und was für Modelle werden erstellt?}
Analyse- und Designentwürfe : diskutieren, abstimmen, dokumentieren und kommunizieren. \\
\begin{itemize}
	\item Verstehen eines Gebildes
	\item  Kommunizieren
	\item Gedankliches Hilfsmittel
	\item Kritisieren
	\item Experimentieren
	\item Aufstellen und Prüfen von Hypothesen
	\item in OOP: 
	\begin{itemize}
		\item Statische Modelle:
		\begin{itemize}
			\item Klassen und Assoziationen
		\end{itemize}
		\item Dynamische Modelle:
		\begin{itemize}
			\item Abläufe und Verhalten
		\end{itemize}
	\end{itemize}
\end{itemize}



\subsection{Welche Artefakte werden in der Anforderungsanalyse erstellt und wozu werden sie gebraucht?}

\begin{itemize}
	\item Systemabgrenzung und Systemkontextdiagramm
	\item Use-Case-Modell und UI-Sketches
	\item Qualitätsanforderungen und Randbedingungen
	\item Domänenmodell
\end{itemize}




\section{LE02}

\subsection{Was ist Usability und Usability-Engineering?}

\textbf{Usability:} Die Effektivität, Effizienz und Zufriedenheit mit der die adressierten Benutzer ihre Ziele erreichen in ihren spezifischen Kontexten.\\
\textbf{Usability Engineering:} Software entwickeln, die die drei Anforderungen von Usability erfüllen.


\subsection{Was ist Usability-Engineering und was sind seine Ziele?}
\begin{itemize}
	\item Usability-Engineering = Software-Ergonomie
	\item Ziel: SW-Produkte entwickeln, die effektiv, effizient und zufriedenstellend sind.
\end{itemize}


\subsection{Welche 7 Usability-Aspekte sind gemäss ISO EN 9241-110 wichtig und was fordern sie?}

\begin{enumerate}
	\item Aufgabenangemessenheit
	\begin{itemize}
		\item Aufwand im Vergleich zu Aufgaben und Ziele sollte angemessen sein.
	\end{itemize}
	\item Selbstbeschreibungsfähigkeit
		\begin{itemize}
		\item Wissen wo in der SW man ist und was man tun muss/kann und was das System tut.
	\end{itemize}
	\item Kontrolle
		\begin{itemize}
		\item Kontrolle über Interaktion mit System haben.
	\end{itemize}
	\item Erwartungskonformität
		\begin{itemize}
		\item Funktionalität
		\item Interaktion
		\item Design
		\item Struktur
		\item Ansprechen der Komplexität
	\end{itemize}
	\item Fehlertoleranz
		\begin{itemize}
		\item Fehler vermeiden
		\item Fehler und Ursache erkennen
		\item Fehler korrigieren
	\end{itemize}
	\item Inidividualisierbarkeit
		\begin{itemize}
		\item Anpassbar auf Bedürfnisse (Laien, Experten, Benutzer mit besondeen Bedürfnisse)
	\end{itemize}
	\item Lernförderlichkeit
		\begin{itemize}
		\item Informationen über unterliegende Konzepte, Reglen, Verfahren und neue Funktionalitäten/Interaktionsmöglichkeiten
	\end{itemize}
\end{enumerate}





















































































































































































































































































































































\end{document}