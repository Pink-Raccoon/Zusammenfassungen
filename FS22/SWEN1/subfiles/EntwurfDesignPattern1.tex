\documentclass[../ZF_SWEN1.tex]{subfiles}

\begin{document}


\subsection{Liste der Design Patterns für die Lerneinheit}
\begin{itemize}
	\item Adapter
	\item Simple Factory 
	\item Singleton
	\item Dependency Injection
	\item Proxy 
	\item Chain of Responsibility
\end{itemize}

\subsubsection{Adapter}

\begin{itemize}
	\item \textbf{Problem}
	\begin{itemize}
		\item Einsatz einer Klasse ist inkompatibel mit bereits definierten domänenspezifischen Interface
	\end{itemize}
	\item \textbf{Lösung}
	\begin{itemize}
		\item Eigene Adapter Klasse dazwischen schalten
	\end{itemize}
	\item \textbf{Hinweise}
	\begin{itemize}
		\item Oft so ein externer Dienst in eigene Anwendung integriert, insbesondere wenn Dienst austauschbar sein soll
		\item Target Interface bewusst für Domänenlogik optimiert, während Adaptee von extern bezogen wird.
		\item Falls Adaptee einen externen Dienst, dann im Adapter allenfalls Kommunikation integrieren.
	\end{itemize}
\end{itemize}


\subsubsection{Simple Factory}
\begin{itemize}
	\item \textbf{Problem}
	\begin{itemize}
		\item Das Erzeugen eines neuen Objekts ist aufwändig
	\end{itemize}
	\item \textbf{Lösung}
		\item Eine eigene Klasse für das Erzeugen eines neuen Objekts wird geschrieben
	\item \textbf{Hinweise}
	\begin{itemize}
		\item Erzeugung oft abhängig von Konfiguration
		\item Auch möglich create() mit Parametern zu ergänzen
		\item Delegation Implementieren von Interfaces z.B
		\item Factory kann allenfalls erzeugten Objekte zwischenspeichern und wiederverwenden
	\end{itemize}
	\item \textbf{Alternativen}
	\begin{itemize}
		\item GoF Abstract Factory: Erzeugung einer Familie von verwandten Objekte
		\item GoF Factory Method: Basisklasse abstrakte Methode definiert, Objekt eines bestimmten Interfaces zu erzeugen. In abgeleiteten Klassen Methode überschrieben und erzeugt gewünschtes Objekt.
	\end{itemize}
\end{itemize}







\subsubsection{Singleton}
\begin{itemize}
	\item \textbf{Problem}
	\begin{itemize}
		\item Benötigt von einer Klasse eine einzige Instanz
		\item Instanz global sichtibar sein
	\end{itemize}
	\item \textbf{Lösung}
	\begin{itemize}
		\item Klasse mit statischen Methode, liefert immer dasselbe Objekt zurück
		\item Statische Methode public deklarieren
	\end{itemize}
	\item \textbf{Hinweise}
	\begin{itemize}
		\item Globale Sichtbarkeit kritisch
		\item Lazy Creation für Instanz möglich, dann aber getInstance() synchronisiert werden.
	\end{itemize}
	\item \textbf{Allgemein}
	\begin{itemize}
		\item Singletons dann wichtig, wenn einen zentralen Ort braucht um Ressourcen zu verwalten
		\item Speicherplatz wird gespart mit nur einem Objekt
		\item Java Enum Instanzen sind ebenfalls Singletons
		\item Globale Sichtbarkeit eher problematisch
	\end{itemize}
\end{itemize}



\subsubsection{Dependency Injection}
\begin{itemize}
	\item \textbf{Problem}
	\begin{itemize}
		\item Klasse braucht Referenz auf ein anderes Objekt. Dieses Objekt muss ein bestimmtes Interface definieren, je nach Konfiguration mit einer anderen Funktionalität.
	\end{itemize}
	\item \textbf{Lösung}
	\begin{itemize}
		\item Anstelle Klasse abhängige Objekt selber erzeugt, Objekt von aussen (Injector) gesetzt
	\end{itemize}
	\item \textbf{Hinweise}
	\begin{itemize}
		\item Ersatz für Factory Pattern
		\item Direkter Widerspruch zu GRASP Creator Prinzip
		\item Unterstüzt von vielen Frameworks kann auch ohne angewendet werden
		\item Erleichtert schreiben von Testfällen insbesondere Gebrauch von Mocks
	\end{itemize}
	\item Varianten
	\begin{itemize}
		\item DI über setter Methoden
		\item Service bei Constructor vom Client übergeben werden
		\item DI initialisiert ganze Anwendung
	\end{itemize}
	\item Frameworks verwenden Annotationen um anzuzeigen welche Attribute über DI gesetzt werden sollen.
	
\end{itemize}




\subsubsection{Proxy}
\begin{itemize}
	\item \textbf{Problem}
	\begin{itemize}
		\item Objekt ist nicht oder noch nicht im selben Adressraum verfügbar
	\end{itemize}
	\item \textbf{Lösung}
	\begin{itemize}
		\item Stellvertreter Objekt mit demselben Interface anstelle des richtigen Objekts verwendet
		\item Proxy Objekt leitet alle Methodenaufrufe zum richtigen Objekt weiter
	\end{itemize}
	\item Einsatz als (Struktur ist dieselbe!)
	\begin{itemize}
		\item Remote Proxy = Proxy für Objekt in einem anderen Adressraum und übernimmt Kommunikation mit diesem
		\item Virtual Proxy = verzögert Erzeugen des richtigen Objekts auf das 1. Mal, dass dieses benutzt wird
		\item Protection Proxy = Kontrolliert Zugriff auf das richtige Objekt
	\end{itemize}
	\item \textbf{Hinweise}
	\begin{itemize}
		\item Unterschied zu Adapter: "Adaptee" in diesem Fall dasselbe Interface implementiert wie "Adapter"
		\item Vom Aufbau identisch mit Decorator Pattern hat aber einen anderen Zweck
		\item Persistenzframeworks verwenden Virtual Proxy Objekte, um das Erzeugen von Objekten und damit das Herunterladen der Daten zu verzögern, bis die Daten wirklich gebraucht werden.
	\end{itemize}
\end{itemize}


\subsubsection{Chain of Responsibility}
\begin{itemize}
	\item \textbf{Problem}
	\begin{itemize}
		\item Für Anfrage potentiell mehrere handler, aber richtigen Handler im Vornherein nicht möglich herauszufinden
	\end{itemize}
	\item \textbf{Lösung}
	\begin{itemize}
		\item Handler in einer einfach verketteten Liste hintereinandergeschachtelt.
		\item Jeder Handler entscheidet, ob der die Anfrage selber beantworten möchte oder sie an den nächsten Handler weiterleitet.
	\end{itemize}
	\item \textbf{Hinweise}
	\begin{itemize}
		\item Variante davon:Jeder Handler leitet Anfrage an den nächsten, unabhängig dabon, ob er sie selber behandelt oder nicht
		\item Könnte auch sein, dass gar kein Handler Anfrage behandelt		
	\end{itemize}
	\item\textbf{ Allgemein}
	\begin{itemize}
		\item Bei allen hierarchischen Stufen: Wenn Referenz auf Parent Node vorhanden, Anfrage zum Parent weiterleiten wenn dies Node selber nicht beabeiten kann.
	\end{itemize}
\end{itemize}

















































































































\end{document}