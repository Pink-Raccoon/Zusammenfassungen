\documentclass[../ZF_HM2.tex]{subfiles}

\begin{document}

\subsection{Problemstellung ODE}

\colorbox{blue!30}{$y^n(x)=$}$f(x,y(x),y'(x),...,y^{(n-1)}(x))$\\
\colorbox{blue!30}{gesucht:} $y=y(x) \implies$ Lösungen im Intervall [a,b]\\
\textbf{\colorbox{magenta!30}{Bemerkungen:}}\\
\begin{enumerate}
	\item \textcolor{magenta}{Neben den ODE gibt es noch die partiellen}
	\item \textcolor{magenta}{Es gibt Systeme von Diff.Gleichungen, die u.Umständen gekoppelt sind, z.B gekoppelte Schwingungen}
	\item \textcolor{magenta}{Allgemeine Lösung: DGL n-ter Ordnung enthält noch n unabhängige Parameter (Integrationskonstante unbestimmter Integrale)}
	\item 
	\textcolor{magenta}{Unterscheidung Anfangs- oder Randwertproblem, bei Anwendung numerischer Verfahren}
\end{enumerate}

\textbf{\colorbox{violet!30}{Definition Anfangswertproblem (AWP)}}\\
\textbf{\underline{1.Ordnung:}}\\
\colorbox{pink!30}{Gesucht:} Spezifische Lösungskurve $y=y(x)$, durch vorgegebenen $P=(x_0,y(x_0))$\\
\colorbox{green!30}{Gegeben:} $y'(x), = f(x,y(x))$ und Anfangswert $y(x_0)$\\
\textbf{\underline{2.Ordnung:}}\\
\colorbox{pink!30}{Gesucht:} Spezifische Lösungskurve $y=y(x)$, durch vorgegebenen $P=(x_0,y(x_0))$ und im Punkt $x_0$ vorgegebene Steigung $y'(x_0)=m$\\
\colorbox{green!30}{Gegeben:} $y''(x), = f(x,y(x),y'(x))$ und Anfangswerte $y(x_0),y'(x_0)$\\

\subsection{Richtungsfelder DGL 1.Ordnung}

\textcolor{teal}{Linienelement:} Pfeil.\\
Tangente $y'(x) = f(x,y(x))$ durch Pfeil graphisch anzeigen, diese geben in jedem Punkt, die Richtung der Lösungskurve an.\\
Idee: Lösung AWP dem Richtungsfeld möglichst genau zu folgen $\implies$ Diskretisierung benötigt:\\
Intervall [a,b] = $[x_i,x_{i+1}]$\\
h = $\dfrac{b-a}{n}$\\
$x_i = a+i*h$\\
$x_{i+1}= x_i + h$\\
$y_{i+1} = y_i + Steigung$

\subsection{Eulerverfahren}
\colorbox{violet!30}{$y'(x) = f(x,y(x))$, $y_0 = y(a)$, $x_0=a$}

\begin{mdframed}
\textbf{Algorithmus:\\}
\underline{Geg AWP:}\\
$y'(x) = f(x,y(x))$, $y_0 = y(a)$\\
\underline{Verfahren:}\\
$x_{i+1}= x_i + h$\\
$y_{i+1} = y_i + h* f(x_i,y_i)$

\end{mdframed}

\subsection{Mittelpunktverfahren}
\begin{mdframed}
\textbf{Algorithmus:\\}
\underline{Geg AWP:}\\
$y'(x) = f(x,y(x))$, $y_0 = y(a)$\\
\underline{Verfahren:}\\
$x_{h/2}= x_i + \dfrac{h}{2}$\\
$y_{h/2}= y_i + \dfrac{h}{2}*f(x_i,y_i)$\\
$x_{i+1}= x_i + h$\\
$y_{i+1} = y_i + h* f(x_{h/2},y_{h/2})$

\end{mdframed}


\subsection{Das modifizierte Eulerverfahren}
\begin{mdframed}
\textbf{Algorithmus:\\}
\underline{Geg AWP:}\\
$y'(x) = f(x,y(x))$, $y_0 = y(a)$\\
\underline{Verfahren:}\\
$x_{i+1}= x_i + h$\\
$yEuler_{i+1} = y_i + h *f(x_i,y_i)$\\
$k1 = f(x_i,y_i)$\\
$k2 = f(x_i, yEuler_{i+1})$\\
$y_{i+1}  = y_i + h*(\dfrac{k1+k2}{2})$

\end{mdframed}

\subsection{Fehlerordnung eines Verfahrens}

\textcolor{violet}{Lokaler Fehler:} Fehler nach einem Verfahren\\
\colorbox{yellow!30}{$\varphi(x_i,h)$:=$y(x_{i+1})-y_{i+1}$}\\
\colorbox{green!30}{Konsistenzordung p} falls:\\
$\varphi(x_i,h) \leq C * h^{p+1} \implies$ genügend kleine h und C$>$0\\\\
\textcolor{violet}{Globaler Fehler:} Gesamtfehler also nach n-Iterationen\\
\colorbox{yellow!30}{$y(x_n)-y_n$}\\
\colorbox{green!30}{Konvergenzordung p} falls:\\
$|y(x_n)-y_n| \leq C*h^p \implies$genügend kleine h und C$>$0\\
\begin{itemize}
	\item Für die hier betrachteten Verfahren: Konsistenzordung = Konvergenzordung
	\item Verwendbar: Nur Verfahren mit Konvergenzordung p $\geq$ 1 $\implies$ Globaler Fehler gegen 0.
\end{itemize}
Eulerverfahren: \textcolor{teal}{$p=1$}\\
Mittelpunktsregel: \textcolor{teal}{$p=2$}\\
Mod.Eulerverfahren: \textcolor{teal}{$p=2$}\\

\subsection{Das klassische 4-Stufige Runge-Kutta Verfahren}
\begin{mdframed}
\textbf{Algorithmus:\\}
\underline{Geg AWP:}\\
$y'(x) = f(x,y(x))$, $y_0 = y(a)$\\
\underline{Verfahren:}\\
$x_{i+1}= x_i + h$\\
$k1 = f(x_i,y_i)$\\
$k2 = f(x_i+\dfrac{h}{2}, y_{i+1}+\dfrac{h}{2})$\\
$k3 = f(x_i+\dfrac{h}{2}, y_{i+1}+\dfrac{h}{2}*k2)$\\
$k4 = f(x_i+h, y_{i+1}+h*k3)$\\
$y_{i+1} = y_i + h*\dfrac{1}{6}(k1+2k2+2k3+k4)$

\end{mdframed}

\subsubsection{Das allgemeine s-stufige Runge-Kutta Verfahren}
K\colorbox{pink!30}{$_n$} = $f(x_i + c_n*h,y_i +h*\sum_{m=1}^{n-1}a_{nm}k_m)$ \colorbox{pink!30}{n=1,...,s}\\
$y_{i+1} = y_i +h*\sum_{n=1}^{s}b_nk_n$\\
s = Stufenzahl, $a_{nm}$,$b_n$,$c_n$ = Konstanten

\begin{table} [H]
\begin{tabular}{l|l|l|l|l|l}
$c_1$ & \\
$c_2$ & $a_{21}$\\
$c_3$ & $a_{31}$ & $a_{32}$ \\
.&.&.\\
.&.&.\\
.&.&.\\
$c_n$ & $a_{n1}$& $a_{n2}$&...&$a_{n,n-1}$\\
.&.&.&...&.\\
.&.&.&...&.\\
.&.&.&...&.\\
$c_s$ & $a_{s1}$& $a_{s2}$&...&$a_{s,s-1}$ \\
\hline
&$b_1$&$b_2$&...&$b_{s-1}$&$b_s$
\end{tabular}

\end{table}
\textcolor{pink}{Euler-Verfahren,s=1}
\begin{table} [H]
\begin{tabular}{l|l}
0 & \\
\hline
&1
\end{tabular}
\end{table}
\textcolor{pink}{Mittelpunkt-Verfahren,s=2}
\begin{table} [H]
\begin{tabular}{l|l|l}
0 \\
0.5&0.5\\
\hline
&0&1
\end{tabular}
\end{table}
\textcolor{pink}{Modifiziertes-Verfahren,s=2}
\begin{table} [H]
\begin{tabular}{l|l|l}
0 \\
1&1\\
\hline
&0.5&0.5
\end{tabular}
\end{table}
\textcolor{pink}{Klass.Runge-Kutta Verfahren, s=4}
\begin{table} [H]
\begin{tabular}{l|l|l|l|l}
0 \\
0.5&0.5\\
0.5&0&0.5\\
1&0&0&1\\
\hline
&$\dfrac{1}{6}$&$\dfrac{1}{3}$&$\dfrac{1}{3}$&$\dfrac{1}{6}$
\end{tabular}
\end{table}

\begin{lstlisting}[language=python]
def interpolate_runge_kutta_custom(f, x, h, y0):
    s = 4

    a = np.array([
        [0, 0, 0, 0],
        [1, 0, 0, 0],
        [1, 1, 0, 0],
        [0.75, 0.5, 0.75, 0]
    ], dtype=np.float64)
    b = np.array([0.1, 0.1, 0.4, 0.4], dtype=np.float64)
    c = np.array([0.75, 0.25, 0.75, 0.5], dtype=np.float64)

    y = np.full(x.shape[0], 0, dtype=np.float64)
    y[0] = y0

    for i in range(x.shape[0] - 1):
        k = np.full(s, 0, dtype=np.float64)

        for n in range(s):
            k[n] = f(x[i] + (c[n] * h), y[i] + h * np.sum([a[n][m] *
             k[m] for m in range(n - 1)]))

        y[i + 1] = y[i] + h * np.sum([b[n] * k[n] for n in range(s)])

    return y

\end{lstlisting}


\subsection{Zurückführen DGL k-ter Ordnung auf k DGL 1.Ordnung}
\begin{mdframed}
\underline{Beispiel (3.Ordnung):}\\
$y'''+5y''+8'y+6y = 10e^{-x}$\\
\underline{AWP:}\\
$y(0)=2, y'(0)= y'(0)=0$\\\\
\underline{1.Schritt:}\\
Nach höchsten Ableitung auflösen:\\
$y''' = 10e^{-x}-5y''-8y'-6y$\\\\
\underline{2.Schritt:}\\
Hilfsfunktionen bis (höchsten-1)Ableitung:\\
$z_1(x)=y(x)$\\
$z_2(x)=y'(x)$\\
$z_3(x)=y''(x)$\\\\
\underline{3.Schritt:}\\
Hilfsfunktionen ableiten und in $z'_3=y'''$ einsetzen:\\
$z_1'(x)=y(x)=(z_2)$\\
$z_2'(x)=y''(x)=(z_3)$\\
$z_3'(x)=y'''(x)$\\
$z_3'(x) = 10e^{-x}-5z_3-8z_2-6z_1$\\\\
\underline{4.Schritt:}\\
DGL in vektorieller form:\\\\
$z'= \left[\begin{matrix}
z_1'\\
z_2'\\
z_3'
\end{matrix} \right]$ = $\left[\begin{matrix}
z_2\\
z_3\\
10e^{-x}-5z_3-8z_2-6z_1
\end{matrix}\right]$ = $f(x,z)$\\
$z(0) = \left[\begin{matrix}
2\\
0\\
0
\end{matrix} \right]$ = $\left[\begin{matrix}
y(0)\\
y'(0)\\
y''(0)
\end{matrix} \right]$\\
weil DGL 3.Ord, als LGL schreiben möglich: $z'=Az + b$\\
A=$\left[\begin{matrix}
0&1&0\\
0&0&1\\
-6&-8&-5
\end{matrix} \right]$, b= $\left[\begin{matrix}
0\\
0\\
10e^{-x}
\end{matrix} \right]$
\end{mdframed}
\textbf{Beispiel Eulerverfahren:\\}
\begin{lstlisting}[language=python]
a = 0.
b = 1.
h = 0.1
n = np.int((b-a)/h)
rows = 4

x = np.zeros(n+1)
z = np.zeros([rows,n+1])


x[0] = a
z[:,0] =np.array([0.,2.,0.,0.])

def f(x,z): 
    return np.array([z[1], z[2], z[3], np.sin(x)+5-1.1*z[3]+0.1*z[2]+0.3*z[0]])
	
for i in range(x.shape[0]-1):
    x[i+1]=x[i]+h
    k1 = f(x[i], z[:,i]) 
    k2 = f(x[i] + (h / 2.0), z[:,i] + (h / 2.0) * k1)
    k3 = f(x[i] + (h / 2.0), z[:,i] + (h / 2.0) * k2)
    k4 = f(x[i] + h, z[:,i] + h * k3)
    z[:,i+1] = z[:,i] + h*(1/6)*(k1+2*k2+2*k3+k4)

\end{lstlisting}

\subsection{Lösen eines Systems von k DGL 1.Ordnung}
Hier $y^{(i)}$ = Vektor nach i-ten Iteration!!!
\begin{mdframed}
\underline{\textbf{Rezept Lösungsverfahren:}}\\
\colorbox{pink!30}{Beispiel}\\
$x_{i+1}= x_i + h$\\
$y^{(i+1)} = y^i + h* f(x_i,y^{(i)})$\\
$y''' = 10e^{-x}-5y''-8y'-6y$\\\\
\colorbox{pink!30}{System}\\
$z'= \left[\begin{matrix}
z_1'\\
z_2'\\
z_3'
\end{matrix} \right]$ = $\left[\begin{matrix}
z_2\\
z_3\\
10e^{-x}-5z_3-8z_2-6z_1
\end{matrix}\right]$ = $f(x,z)$\\
$z(0) = \left[\begin{matrix}
2\\
0\\
0
\end{matrix} \right]$ = $\left[\begin{matrix}
z_1^{(0)}\\
z_2^{(0)}\\
z_3^{(0)}
\end{matrix} \right]$\\\\
\colorbox{violet!30}{i=0:}\\
 $f(x_0,z^{(0)})=\left[\begin{matrix}
z_2^{(0)}\\\\
z_3^{(0)}\\\\
10e^{-x_0}-5z_3^{(0)}-8z_2^{(0)}-6z_1^{(0)}
\end{matrix}\right]$ = $\left[\begin{matrix}
0\\
0\\
-2
\end{matrix} \right]$\\
\colorbox{blue!30}{$x_1= x_0 + h = 0.5$}\\
\colorbox{blue!30}{$z^{(1)}= z^{(0)} + h *f(x_0,z^{(0)}) = \left[\begin{matrix}
2\\
0\\
0
\end{matrix} \right] + 0.5\left[\begin{matrix}
0\\
0\\
-2
\end{matrix} \right]= \left[\begin{matrix}
2\\
0\\
-1
\end{matrix} \right]$}\\\\\\\\\\\\
\colorbox{violet!30}{i=1:}\\
 $f(x_0,z^{(0)})=\left[\begin{matrix}
z_2^{(1)}\\\\
z_3^{(1)}\\\\
10e^{-x_0}-5z_3^{(1)}-8z_2^{(1)}-6z_1^{(1)}
\end{matrix}\right]$ = $\left[\begin{matrix}
0\\
-1\\
0.9347
\end{matrix} \right]$\\
\colorbox{blue!30}{$x_2= x_1 + h = 1$}\\
\colorbox{blue!30}{$z^{(2)}= z^{(1)} + h *f(x_1,z^{(1)}) = \left[\begin{matrix}
2\\
0\\
-1
\end{matrix} \right] + 0.5\left[\begin{matrix}
0\\
-1\\
0.9347
\end{matrix} \right]= \left[\begin{matrix}
2\\
-0.5\\
-1.4673
\end{matrix} \right]$}\\\\
\end{mdframed}

\begin{lstlisting}[language=python]
a = 0
b = 60
h = 0.1
n = int((b-a)/h)
c = 0.16
m = 1
l = 1.2
g = 9.81

rows = 2
phi0 = np.pi/2
x = np.zeros(n+1)
z = np.zeros([rows,n+1])
x[0] = a
z[:,0] =np.array([phi0,0],dtype=np.float64)


def f(x,z):
    return np.array([z[1],-((c/m)*z[1])-(g/l)*np.sin(z[0])])
	
for i in range(x.shape[0] - 1):
    x[i+1]=x[i]+h
    k1 = f(x[i], z[:,i])
    k2 = f(x[i] + (h / 2.0), z[:,i] + (h / 2.0) * k1)
    k3 = f(x[i] + (h / 2.0), z[:,i] + (h / 2.0) * k2)
    k4 = f(x[i] + h, z[:,i] + h * k3)

    z[:,i+1] = z[:,i] + h * (1 / 6.0) * (k1 + 2 * k2 + 2 * k3 + k4)

\end{lstlisting}













\end{document}