\documentclass[../ZF_HM2.tex]{subfiles}

\begin{document}



\subsection{Ausgleichsrechnung}
\begin{itemize}
	\item Datenpunkte mit gewissen Streuung durch einfache Funktion annähern
	\item Mehr Gleichungen als unbekannte (Mehr Datenpunkte als Parameter)

\end{itemize}

\subsubsection{Polynominterpolation}
\textcolor {magenta} {Gesucht: $P_n(x)$ welche n+1 Stützpunkte interpoliert}
Jeder Stützpunkt gibt lin.Gleichung für die \colorbox {orange!30}{Bestimmung der Koeffizienten.}\\
\colorbox {green!30}{Grad n} so wählen , dass lin.Gleichungssystem gleich viele Gleichungen wie unbekannte Koeffizienten hat.\\
$P_n(x_0) = a_0 + a_1x_0 + a_2x_0^{2} + ... + a_nx_0^{n}=y_0$\\\\
$P_n(x_1) = a_0 + a_1x_1 + a_2x_1^{2} + ... + a_nx_1^{n}=y_1$\\
		.\\
		.\\
		.\\
$P_n(x_n) = a_0 + a_1x_n + a_2x_n^{2} + ... + a_nx_n^{n}=y_n$\\\\


\[
\begin{rcases*}
 = \left[\begin {matrix}1 & x_0 & ... & x_0^{n}\\
.&.\\
.&.\\
.&.\\
1 & x_n &...& n_n^{n}\end{matrix}\right] *
 \left[\begin {matrix}a_0\\
.\\
.\\
.\\
a_n\end{matrix}\right]
= \left[\begin {matrix}y_0\\
.\\
.\\
.\\
y_n\end{matrix}\right] 
\end{rcases*}\textcolor {orange!60}{Van der monde-Matrix = Schlecht konditioniert.
Ab n \geq 20 numerisch instabil}
\]


\begin{mdframed}
	\textbf{Lagrange Interpolationsformel:\\}
	Lagrangeform von $P_n(x):$\\
	 \[ P_(x) =\sum_{i=0}^{n} l_i(x)y_i \]
	
	LagrangePolynome = $l_i(x)$:
	\[ l_i(x) =\prod_{j=0_{j\neq i}}^{n} \dfrac {x-x_j}{x_i-x_j} \]
	
\end{mdframed}

\textbf{Stückweise Interpolation:\\}
Interpolationspolynom erster Ordnung: n = 1\\
Stützpunkte so wählen: $x_{j-1} und x_{j+1}$ ,somit 2 Werte: n = 1\\

\textbf{Grösse des Fehlers an Stelle x wenn:\\}
$y_i$ Funktionswerte (genügend oft stetig differenzierbare Funktion)\\
$|f(x)-P_n(x)|\leq \dfrac{|(x-x_0)(x-x_1)...(x-x_n)|}{(n+1!)}$\\
Max der (n+1)-ten Abletung der f(x) Intervall [$x_0,x_n$] kennen, Fehlerabschätzung nur dann möglich.


\subsubsection{Spline-Interpolation}

\textbf{Bedingungen für $S_i$:\\}

\begin{enumerate}
	\item $S_i(x_i) = y_i, S_{i+1}(x+1) = y_{i+1}, ... \textcolor{green} {Interpolation}$
	\item $S_i(x_{i+1}) = S_{i+1}(x_{i+1}),S_{i+1}(x_{i+2}) = S_{i+2}(x_{i+2}),...$\textcolor {green} {Stetiger Übergang}
	\item $S'_i(x_{i+1}) = S'_{i+1}(x_{i+1}),S'_{i+1}(x_{i+2}) = S'_{i+2}, ... \textcolor{green}{Keine Knicke}$
		\item $S''_i(x_{i+1}) = S''_{i+1}(x_{i+1}),S''_{i+1}(x_{i+2}) = S''_{i+2}, ...$ \textcolor{green}{Gleiche Krümmung}
		\item Mindestens den Grad 3 ... \textcolor{green}{Kubische Splines}
\end{enumerate}



3 Intervalle, 4 Stützpunkte:\\
$[x_0,x_1],[x_1,x_2],[x_2,x_3]$\\
\textbf{Ansatz:\\}
$S_0 = a_0 + b_0(x-x_0) + c_0(x-x_0)^{2} + d_0(x-x_0)^{3},x\in[x_0,x_1]$\\
$S_1 = a_1 + b_1(x-x_1) + c_1(x-x_1)^{2} + d_1(x-x_1)^{3},x\in[x_1,x_2]$\\
$S_2 = a_2 + b_2(x-x_2) + c_2(x-x_2)^{2} + d_2(x-x_2)^{3},x\in[x_2,x_3]$\\

$3*4 = 12$ Koeffizienten$\implies $12 Bedingungen

\textbf{\textcolor{blue}{1.Interpolation der Stützpunkte:}\\}
$1. S_0(x_0) = y_0$\\
$2. S_1(x_1) = y_1$\\
$3. S_2(x_2) = y_2$\\
$4. S_3(x_3) = y_3$\\

\textbf{\textcolor{blue}{2. Stetiger Übergang an Stellen $x_1$ und $x_2$:}\\}
$5. S_0(x_1) =S_1(x_1)$\\
$6. S_1(x_2) =S_2(x_2)$\\

\textbf{\textcolor{blue}{3. Erste Ableitung an Übergangstellen übereinstimmen:}\\}
$7. S'_0(x_1) =S'_1(x_1)$\\
$8. S'_1(x_2) =S'_2(x_2)$\\

\textbf{\textcolor{blue}{4. Zweite Ableitung an Übergangstellen übereinstimmen:}\\}
$9. S''_0(x_1) =S''_1(x_1)$\\
$10. S''_1(x_2) =S''_2(x_2)$\\

$= 10 Bedingungen$\\

Die weiteren 2 Bedingungen können "frei gewählt" werden.\\
Diese beziehen sich häufig auf Randstellen $x_0$ und $x_3$.\\
\textbf{Beispiele:\\}
\textcolor{violet}{Natürliche kubische Splinefunktion:}\\
Mit moeglichen Wendepunkt im Anfangs und Endpunkt.\\
\[
\begin{rcases*}
S'''_0(x_0) = 0\\
S'''_2(x_3) = 0\\
\end{rcases*} 
\]
\textcolor{violet}{Periodische kubische Splinefunktion:\\}
Wenn man Periode $p=x_3-x-0$ hat und damit $y_0$ bzw. $S_0(x_0) = S_2(x_3)$ gilt
\[
\begin{rcases*}
S'_0(x_0) = S'_2(x_3) \\
S''_0(x_0) = S''_2(x_3) \\
\end{rcases*} 
\]
\textcolor{violet}{mit not-a-knot Bedingung kubische Splinefunktion:\\}
s.d Auch dritte Ableitung in $x-1,x-2$ noch stetig ist. ($x_1,x_2$ keine echten Knoten)
\[
\begin{rcases*}
S'''_0(x_1) = S'''x_1(x_1) \\
S'''_1(x_2) = S'''_2(x_2) \\
\end{rcases*} 
\]


\begin{mdframed}
	\textbf{Algorithmus: natürliche kubische Splinefunktion\\}
	$S_i = a_i + b_i(x-x_i) + c_i(x-x_i)^{2} + d_i(x-x_i)^{3}$\\
Koeffizienten $a_i,b_i,c_i,d_i$ berechnen:
\begin{enumerate}
	\item $a_i = y_i$
	\item $h_i = x_{i+1} = -x_i$
	\item $c_0 = 0, c_n=0$ \textcolor{red}{!!! (für periodisch$(s_1 = s_0)$,für not a knot$(d_0 = d_1 d_{n-2} = d_{n-1}$)}
	\item Berechnung der Koeffizienten $c_1,c_2,c_3,...,c_{n-1}$
	\begin{itemize}
		\item \colorbox {green!30} {i = 1:}
		\begin{itemize}
			\item \colorbox{yellow!30}{$2(h_0 + h_1)$} \colorbox {violet!30}{$c_1$} + \colorbox{yellow!30}{$h_1$}\colorbox {violet!30}{$c_2$} = \colorbox{orange!30}{$3\dfrac{y_2-y_1}{h_1}-3\dfrac{y_1-y_0}{h_0}$}
		\end{itemize}
		\item falls \colorbox {green!30} {$n \geq 4$} dann für i = 2,...,n-2
		\begin{itemize}
			\item \colorbox{yellow!30}{$h_{i-1}$}\colorbox {violet!30}{$c_{i-1}$} + \colorbox{yellow!30}{$2(h_{i-1} + h_i)$}\colorbox {violet!30}{$c_{i}$} + \colorbox{yellow!30}{$h_i$}\colorbox {violet!30}{$c_{i+1}$} = \colorbox{orange!30}{$3\dfrac{y_{i+1}-y_i}{h_i}-3\dfrac{y_i-y_{i-1}}{h_{i-1}}$}
		\end{itemize}
		\item \colorbox {green!30} {i = n - 1:}
		\begin{itemize}
			\item \colorbox{yellow!30}{$2(h_{n-2} + h_i)$}\colorbox {violet!30}{$c_{n-2}$} + \colorbox{yellow!30}{$2(h_{n-2} + h_{n-1})$}\colorbox {violet!30}{$c_{n-1}$} = \colorbox{orange!30}{$3\dfrac{y_{n}-y_{n-1}}{h_{n-1}}-3\dfrac{y_{n-1}-y_{n-2}}{h_{n-2}}$}
		\end{itemize}
		
	\end{itemize}
	\item $b_i$ = $\dfrac{y_{i+1}-y_{i}}{h_{i}}-\dfrac{h_{i}}{3}(c_{i+1}+2c_i)$
	\item $d_i$ = $\dfrac{1}{3h_i}(c_{i+1}-c_i)$
\end{enumerate}
	
\end{mdframed}

\textcolor {magenta}{ Ergibt das Gleichungssystem \colorbox {yellow!30} {A} \colorbox {violet!30} {c} = \colorbox {orange!30} {z}}
\begin{itemize}
	\item \textcolor {magenta} {A ist immer invertierbar} 
	\item \textcolor {magenta} {Numerische Lsg.durch Gauss-Algo} 
	\item \textcolor {magenta} {System ist gut konditioniert} 
	\item \textcolor {magenta} {Pivotsuche nicht erforderlich} 
\end{itemize}
\textit{i = 0,...,n-1}\\
$A = $\colorbox {yellow!30}{$ \left[\begin {matrix} &  &  & \\ 
&  &  &\\
&  &  &\\
&  &  &\\
\end{matrix}\right]$} $c = $\colorbox {violet!30}{$ \left[\begin {matrix}  \\
\\
\\
\end{matrix}\right]$} = $z = $\colorbox {orange!30}{$ \left[\begin {matrix}  \\
\\
\\
\end{matrix}\right]$}


\subsubsection{Lineare Ausgleichsprobleme}
\colorbox {pink!30}{Ausgleichsgerade:} $f(x)=ax+b$:
\colorbox {pink!30} {gesucht:}$F=af_1+bf_2$ $|$ $a,b \in \mathbb{R}$
\textbf{Basisfunktion:} $f_1(x) = x$ und $f_2(x) = 1$



	\textbf{Form Fehlerfunktional:\\\\}
	\colorbox{magenta!30}{$E(f)(a,b):= \sum_{i=1}^{n}(y_i-f(x_i))^2 = \sum_{i=1}^{n}(y_i-(ax_i +b))^2 )$}
	 

\colorbox {pink!30}{minimal = } patiellen Ableitung nach a,b müssen verschwinden

$ \left[\begin {matrix}\sum_{i=1}^{n}(x_i)^2 & \sum_{i=1}^{n}(x_i)\\\\
\sum_{i=1}^{n}(x_i)&n\\
\end{matrix}\right] *
 \left[\begin {matrix}a\\
b\\
\end{matrix}\right]$
=
$ \left[\begin {matrix}\sum_{i=1}^{n}(x_iy_i) \\\\
\sum_{i=1}^{n}(y_i)\\
\end{matrix}\right]$

\textcolor{violet}{Nach a,b auflösen}\\\\
\textbf{Allgemeine Definition:}\\
Gegeben:\\ n-Wertepaare: $(x_i,y_i), i=1,...,n$\\
Basisfunktionen: $(f_1,...,f_m) [a,b]$\\
Ansatzfunktionen: $f:= \lambda_1f_1+...+\lambda_mf_m$\\


\textcolor{pink}{Lineares Ausgleichproblem} mit Fehlerfunktional:\\
$E(f)= \|y-f(x)\|_2^2$ = $\sum_{i=1}^{n}(y_i - \sum_{i=1}^{m}\lambda_j*f_j*(x_i))^2$ = $\|y-A\lambda\|_2^2$\\
\colorbox{violet!30}{Fehlergleichungssystem:} \textcolor{violet}{$A\lambda=y$}\\
A=$ \left[\begin {matrix}
f_1(x_1) & f_2(x)(x_1)&...&f_m(x_1)\\
f_1(x_2) & f_2(x_2)(x_1)&...&f_m(x_2)\\
.&&&\\
.\\
.\\
f_1(x_n) & f_2(x_n)(x_n)&...&f_m(x_n)\\\\
\end{matrix}\right] *
y= \left[\begin {matrix}y_1\\
y_2\\
.\\
.\\
.\\
y_n
\end{matrix}\right]$
= $\lambda
 \left[\begin {matrix}\lambda_1\\
\lambda_2\\
.\\
.\\
.\\
\lambda_n
\end{matrix}\right]$
\\
\textcolor{teal}{Normalgleichungen:} $\dfrac{\delta E(f)(\lambda_1,...,\lambda_m)}{\delta\lambda_j}$ \\\\
\textcolor{teal}{Normalgleichungssystem:}$A^TA\lambda=A^Ty$\\\\
\textcolor{magenta}{Lösung:}$\lambda$ beinhaltet gesuchte Prameter des linearen Ausgleichproblems\\
Verfahren am Besten mit QR-Zerlegung

\subsubsection{Nichtlineare Ausgleichprobleme}
\begin{mdframed}
	\textbf{Allgemeines Ausgleichproblem:\\\\}
	\colorbox{blue!10}{Fehlerfunktional:}\\
	$f_p$ = Ansatzfunktion
	
	$E(f)(a,b):= \sum_{i=1}^{n}(y_i-f_p(\lambda_1,\lambda_2,...,\lambda_m,x_i))^2$
	= $ \left[\begin {matrix}
y_1-f_p(\lambda_1,\lambda_2,...,\lambda_m,x_1)\\
y_2-f_p(\lambda_1,\lambda_2,...,\lambda_m,x_2)\\
.&&&\\
.\\
.\\
y_n-f_p(\lambda_1,\lambda_2,...,\lambda_m,x_n)\\\\\\
\end{matrix}\right]\|_2^2 = \|y-f(\lambda)\|_2^2$\\

$f(\lambda):=f(\lambda_1,\lambda_2,...,\lambda_m):=$ $\left[\begin {matrix}
f_p(\lambda_1,\lambda_2,...,\lambda_m)\\
f_p(\lambda_1,\lambda_2,...,\lambda_m)\\
.&&&\\
.\\
.\\
f_p(\lambda_1,\lambda_2,...,\lambda_m)\\
\end{matrix}\right]$:=
$ \left[\begin {matrix}
f_p(\lambda_1,\lambda_2,...,\lambda_m,x_1)\\
f_p(\lambda_1,\lambda_2,...,\lambda_m,x_2)\\
.&&&\\
.\\
.\\
f_p(\lambda_1,\lambda_2,...,\lambda_m,x_n)\\
\end{matrix}\right]$,\\
y=$\left[\begin{matrix}
y_1\\
y_2\\
.\\
.\\
.\\
y_n
\end{matrix}\right]$, $\lambda=\left[\begin {matrix}\lambda_1\\
\lambda_2\\
.\\
.\\
.\\
\lambda_n
\end{matrix}\right]$
	 
\end{mdframed}
Falls Ansatzfunktionen linear in den Parametern $\implies$\\ Spezialfall des Ausgleichsproblems: $f(\lambda)= A \lambda \implies$ \colorbox{orange!30}{Allg.Ausgleichsproblem}=\colorbox{violet!30}{Minimums einer Funktion}

\begin{mdframed}
	\textbf{Schritte:\\}
	\begin{enumerate}
		\item Normalgleichungen aufstellen (part.Abl.von f)
		\item $\lambda_i =0$
		\item Nichtlineare Gleichungssystem lösen
	\end{enumerate}
	
	
\end{mdframed}
\subsubsection{Das Gauss-Newton Verfahren}
\textcolor {pink!200}{Quadratmittelproblem:} Problem einen Vektor x zu finden für den E(x) minimal wird.\\
\textcolor{violet!30}{Nichtlineare Ausgleichsprobleme:} \colorbox {magenta!30}{$g(\lambda):=y-f(\lambda)$}\\
\textcolor{orange!50}{Fehlerfunktional:}\colorbox {magenta!30}{$E(\lambda):=\|g(\lambda)\|_2^2=\|y-f(\lambda)\|_2^2$}\\
\textcolor{teal!50}{Gauss-Newton-Verfahren:} Lin.Ausgl.RG + Newton-Verfahren = \colorbox{magenta!30}{$g(\lambda)=g(\lambda^{(0)})+Dg(\lambda^{0})*(\lambda - \lambda^{0})$} = Verallg.Tangentgleichung!!\\
$Dg(\lambda^{0})=\left[\begin{matrix}
\dfrac{\delta g_1}{\lambda_1}(\delta\lambda^{0}) & \dfrac{\delta g_1}{\lambda_2}(\delta\lambda^{0}) & ...& \dfrac{\delta g_1}{\lambda_m}(\delta\lambda^{(0)})\\\\
\dfrac{\delta g_2}{\lambda_1}(\delta\lambda^{0}) & \dfrac{\delta g_2}{\lambda_2}(\delta\lambda^{0}) & ...& \dfrac{\delta g_2}{\lambda_m}(\lambda^{(0)})\\\\
.\\
.\\
.\\
\dfrac{\delta g_n}{\lambda_1}(\delta\lambda^{0}) & \dfrac{\delta g_n}{\lambda_2}(\delta\lambda^{0}) & ...& \dfrac{\delta g_n}{\lambda_m}(\delta\lambda^{(0)})\\\\

\end{matrix}\right]$\\
Minimum Fehlerfunktionals: $E(\lambda) = 
\underbrace{\|g(\lambda^{(0)})\|}_\text{\colorbox{pink!30}{y}}+ 
\underbrace{Dg(\lambda^{0})}_\text{\colorbox{pink!30}{-A}}*
\underbrace{\lambda - \lambda^{0})}_\text{\colorbox{pink!30}{$\delta$}}$\\
\colorbox{blue!30}{Normagleichungssystem:}$A^{T}A\delta = A^{T}y = Dg(\lambda^{0})^{T}Dg(\lambda^{0})\delta=Dg(\lambda^{0})^{T}*g(\lambda^{0})$
$\implies$\\ QR-Zerlegung: $Dg(\lambda^{0})$=QR$\geq R\delta = -Q^{T}g(\lambda^{0})$\\
$\implies$ Lösung: $\Lambda = (\lambda^{0})+\delta$
\begin{mdframed}
	\textbf{Schritte:\\}
	\begin{enumerate}
		\item Berechne Funktion
		\begin{itemize}
			\item $g(\lambda):= y-f(\lambda)$
			\item $Dg(\lambda)$
		\end{itemize}
		\item k = 0,1...:\\ $\delta^{k}$ als Lösung: $min\|g(\lambda^{k})+Dg(\lambda^{k})*\delta^{k}\|_2^{2}$\\
		$\implies Dg(\lambda^{k})^{T}*Dg(\lambda^{k})\delta^{k} = Dg(\lambda^{k})^{T}*g(\lambda^{k})$\\
		nach \colorbox{pink!30}{nach $\delta^{k}$ auflösen} durch QR-Zerlegung.
		\begin{itemize}
			\item \colorbox{pink!30}{$Dg(\lambda^{k})=Q^{k}R^{k}$}
			\item \colorbox{pink!30}{$R^{k}\delta^{k}=-(Q^{k})^{T}g(\lambda^{k})$}
		\end{itemize}
		\item Setze: \colorbox{pink!30}{$\lambda^{k+1} = \lambda^{k}+\delta^{k}$}

	\end{enumerate}
\end{mdframed}

Korrektur \colorbox{orange!30}{$\delta^{k}$} nur \colorbox{orange!30}{akzeptiert wenn} Fehlerfunktional zur Abnahme führt:
\textbf{Fehlerfunktional\\}\colorbox{magenta!30}{$E(\lambda^{k+1})=\|g(\lambda^{k+1})\|_2^2 < \|g(\lambda^{k})\|_2^2 = E(\lambda^{k})$}


\begin{mdframed}
	\textbf{Gedämpftes Gauss-Newton-Verfahren}
	\begin{enumerate}
		\item Berechne Funktion:
		\begin{itemize}
			\item  $g(\lambda):= y-f(\lambda)$
			\item $Dg(\lambda)$
			\end{itemize}
		\item k = 0,1...:\\ $\delta^{k}$ als Lösung: $min\|g(\lambda^{k})+Dg(\lambda^{k})*\delta^{k}\|_2^{2}$\\
		$\implies Dg(\lambda^{k})^{T}*Dg(\lambda^{k})\delta^{k} = Dg(\lambda^{k})^{T}*g(\lambda^{k})$\\
		nach \colorbox{pink!30}{nach $\delta^{k}$ auflösen} durch QR-Zerlegung.
		\begin{itemize}
			\item \colorbox{pink!30}{$Dg(\lambda^{k})=Q^{k}R^{k}$}
			\item \colorbox{pink!30}{$R^{k}\delta^{k}=-(Q^{k})^{T}g(\lambda^{k})$}
		\end{itemize}
		\item Finde das minimale p $\in$ {0,1,...,$p_{max}$}\\ $\underbrace{\|\lambda^{(k)}+\dfrac{\delta^{k}}{2^{p}}\|_2^{2}}_\text{\colorbox{pink!30}{$\lambda^{(k+1)}$}}<\|g(\lambda^{(k)})\|_2^{2}$\\
		\item Falls kein min. p gefunden:\\
		p=0 und weiterfahren
		\item Setze: $\lambda^{(k)}+\dfrac{\delta^{(k)}}{2^{p}}$
		
	\end{enumerate}
\end{mdframed}
\colorbox{teal!30}{Abbruchkriterium:}  $\|\dfrac{\delta^{(k)}}{2^{p}}\| < TOL$ \\Keine Grantie, dass die Näherung max. Abstand von TOL zum ges. Min. ist.



\end{document}

