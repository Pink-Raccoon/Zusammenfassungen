\documentclass[../ZF_HM2.tex]{subfiles}

\begin{document}


\subsection{Ausgleichsrechnung}
\begin{itemize}
	\item Datenpunkte mit gewissen Streuung durch einfache Funktion annähern
	\item Mehr Gleichungen als unbekannte (Mehr Datenpunkte als Parameter)

\end{itemize}

\subsubsection{Polynominterpolation}
\textcolor {magenta} {Gesucht: $P_n(x)$ welche n+1 Stützpunkte interpoliert}
Jeder Stützpunkt gibt lin.Gleichung für die \colorbox {orange!30}{Bestimmung der Koeffizienten.}\\
\colorbox {green!30}{Grad n} so wählen , dass lin.Gleichungssystem gleich viele Gleichungen wie unbekannte Koeffizienten hat.\\
$P_n(x_0) = a_0 + a_1x_0 + a_2x_0^{2} + ... + a_nx_0^{n}=y_0$\\\\
$P_n(x_1) = a_0 + a_1x_1 + a_2x_1^{2} + ... + a_nx_1^{n}=y_1$\\
		.\\
		.\\
		.\\
$P_n(x_n) = a_0 + a_1x_n + a_2x_n^{2} + ... + a_nx_n^{n}=y_n$\\\\


\[
\begin{rcases*}
 = \left[\begin {matrix}1 & x_0 & ... & x_0^{n}\\
.&.\\
.&.\\
.&.\\
1 & x_n &...& n_n^{n}\end{matrix}\right] *
 \left[\begin {matrix}a_0\\
.\\
.\\
.\\
a_n\end{matrix}\right]
= \left[\begin {matrix}y_0\\
.\\
.\\
.\\
y_n\end{matrix}\right] 
\end{rcases*}\textcolor {orange!60}{Van der monde-Matrix = Schlecht konditioniert.
Ab n \geq 20 numerisch instabil}
\]


\begin{mdframed}
	\textbf{Lagrange Interpolationsformel:\\}
	Lagrangeform von $P_n(x):$\\
	 \[ P_(x) =\sum_{i=0}^{n} l_i(x)y_i \]
	
	LagrangePolynome = $l_i(x)$:
	\[ l_i(x) =\prod_{j=0_{j\neq i}}^{n} \dfrac {x-x_j}{x_i-x_j} \]
	
\end{mdframed}

\textbf{Stückweise Interpolation:\\}
Interpolationspolynom erster Ordnung: n = 1\\
Stützpunkte so wählen: $x_{j-1} und x_{j+1}$ ,somit 2 Werte: n = 1\\

\textbf{Grösse des Fehlers an Stelle x wenn:\\}
$y_i$ Funktionswerte (genügend oft stetig differenzierbare Funktion)\\
$|f(x)-P_n(x)|\leq \dfrac{|(x-x_0)(x-x_1)...(x-x_n)|}{(n+1!)}$\\
Max der (n+1)-ten Abletung der f(x) Intervall [$x_0,x_n$] kennen, Fehlerabschätzung nur dann möglich.


\subsubsection{Spline-Interpolation}

\textbf{Bedingungen für $S_i$:\\}

\begin{enumerate}
	\item $S_i(x_i) = y_i, S_{i+1}(x+1) = y_{i+1}, ... \textcolor{green} {Interpolation}$
	\item $S_i(x_{i+1}) = S_{i+1}(x_{i+1}),S_{i+1}(x_{i+2}) = S_{i+2}(x_{i+2}),...$\textcolor {green} {Stetiger Übergang}
	\item $S'_i(x_{i+1}) = S'_{i+1}(x_{i+1}),S'_{i+1}(x_{i+2}) = S'_{i+2}, ... \textcolor{green}{Keine Knicke}$
		\item $S''_i(x_{i+1}) = S''_{i+1}(x_{i+1}),S''_{i+1}(x_{i+2}) = S''_{i+2}, ...$ \textcolor{green}{Gleiche Krümmung}
		\item Mindestens den Grad 3 ... \textcolor{green}{Kubische Splines}
\end{enumerate}



3 Intervalle, 4 Stützpunkte:\\
$[x_0,x_1],[x_1,x_2],[x_2,x_3]$\\
\textbf{Ansatz:\\}
$S_0 = a_0 + b_0(x-x_0) + c_0(x-x_0)^{2} + d_0(x-x_0)^{3},x\in[x_0,x_1]$\\
$S_1 = a_1 + b_1(x-x_1) + c_1(x-x_1)^{2} + d_1(x-x_1)^{3},x\in[x_1,x_2]$\\
$S_2 = a_2 + b_2(x-x_2) + c_2(x-x_2)^{2} + d_2(x-x_2)^{3},x\in[x_2,x_3]$\\

$3*4 = 12$ Koeffizienten$\implies $12 Bedingungen

\textbf{\textcolor{blue}{1.Interpolation der Stützpunkte:}\\}
$1. S_0(x_0) = y_0$\\
$2. S_1(x_1) = y_1$\\
$3. S_2(x_2) = y_2$\\
$4. S_3(x_3) = y_3$\\

\textbf{\textcolor{blue}{2. Stetiger Übergang an Stellen $x_1$ und $x_2$:}\\}
$5. S_0(x_1) =S_1(x_1)$\\
$6. S_1(x_2) =S_2(x_2)$\\

\textbf{\textcolor{blue}{3. Erste Ableitung an Übergangstellen übereinstimmen:}\\}
$7. S'_0(x_1) =S'_1(x_1)$\\
$8. S'_1(x_2) =S'_2(x_2)$\\

\textbf{\textcolor{blue}{4. Zweite Ableitung an Übergangstellen übereinstimmen:}\\}
$9. S''_0(x_1) =S''_1(x_1)$\\
$10. S''_1(x_2) =S''_2(x_2)$\\

$= 10 Bedingungen$\\

Die weiteren 2 Bedingungen können "frei gewählt" werden.\\
Diese beziehen sich häufig auf Randstellen $x_0$ und $x_3$.\\
\textbf{Beispiele:\\}
\textcolor{violet}{Natürliche kubische Splinefunktion:}\\
Mit moeglichen Wendepunkt im Anfangs und Endpunkt.\\
\[
\begin{rcases*}
S'''_0(x_0) = 0\\
S'''_2(x_3) = 0\\
\end{rcases*} 
\]
\textcolor{violet}{Periodische kubische Splinefunktion:\\}
Wenn man Periode $p=x_3-x-0$ hat und damit $y_0$ bzw. $S_0(x_0) = S_2(x_3)$ gilt
\[
\begin{rcases*}
S'_0(x_0) = S'_2(x_3) \\
S''_0(x_0) = S''_2(x_3) \\
\end{rcases*} 
\]
\textcolor{violet}{mit not-a-knot Bedingung kubische Splinefunktion:\\}
s.d Auch dritte Ableitung in $x-1,x-2$ noch stetig ist. ($x_1,x_2$ keine echten Knoten)
\[
\begin{rcases*}
S'''_0(x_1) = S'''x_1(x_1) \\
S'''_1(x_2) = S'''_2(x_2) \\
\end{rcases*} 
\]


\begin{mdframed}
	\textbf{Algorithmus: natürliche kubische Splinefunktion\\}
	$S_i = a_i + b_i(x-x_i) + c_i(x-x_i)^{2} + d_i(x-x_i)^{3}$\\
Koeffizienten $a_i,b_i,c_i,d_i$ berechnen:
\begin{enumerate}
	\item 
\end{enumerate}
	
\end{mdframed}


























\end{document}

