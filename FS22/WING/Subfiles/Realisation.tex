\documentclass[../ZF_Wing.tex]{subfiles}

\begin{document}
\textbf{Ziel:} Möglichst effizienter Produktionsprozess (Auslastung, Kapazität, Zeit)
\subsection{Produktionsprogrammplanung}
\begin{itemize}
	\item Produktionsprogramm
	\begin{itemize}
		\item Bestimmt Art, Menge und Zeitpunkt der zu produzierenden Produkte in einem Unternehmen.
	\end{itemize}
	\item Produktionsprogrammbreite
	\begin{itemize}
		\item Anzahl der von einem Unternehmen hergestellten Produktarten
	\end{itemize}
	\item Programmtiefe
	\begin{itemize}
		\item Anzahl der Artikel und Typen, die innerhalb einer Produktart vom Unternehmen angeboten werden.
	\end{itemize}
	\item Idealfall
	\begin{itemize}
		\item Die Ressourcen sind optimal ausgelastet, d.h. Mensch und Maschinen sind weder unterbeschäftigt noch überbeansprucht.
	\end{itemize}
\end{itemize}
Unberbeschäftigung: Hohe Kosten\\
Überbeschäftigung: Burnouts bei Menschen\\

\subsection{Kapazitätsplanung}
\begin{itemize}
	\item Auslastung
	\item Maschinenbelegung
	\item  Produktionsziele
\end{itemize}
\subsection{Fertigungsstrukturen}
\begin{itemize}
	\item Geringe Fertigungstiefe (Grosse Teile der Produktion extern durchgeführt)
	\item Mittlere Fertigungstiefe
	\item Grosse Fertigungstiefe (Die meisten Leistungen selbst erstellt, nur wenige extern)
\end{itemize}



\subsection{Fertigungstypen}
\begin{itemize}
	\item Einmalfertigung
	\item Einzel- und Kleinserienfert.
	\item Serienfertigung
	\item Massenfertigung
\end{itemize}

\subsection{Fertigungsverfahren}
\begin{itemize}
	\item Werkstattprinzip
	\item Fliessprinzip
	\item Gruppenfertigung
\end{itemize}


\subsection{Kennzahlen der Produktion}
Rentabilität : $\dfrac{Ertrag - Aufwand}{Kapitaleinsatz} = \dfrac{Gewinn}{Kapitaleinsatz}$\\\\
Produktivität : $\dfrac{Ausbringungsmenge}{Faktoreinsatzmenge}$\\\\
Wirtschaftlichkeit : $\dfrac{Ertrag}{Aufwand}$\\\\
Produktivität : $\dfrac{Fehlerhafte Produktion}{Total hergestellte Produkte}$\\\\




























\end{document}